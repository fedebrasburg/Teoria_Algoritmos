\documentclass[a4paper,10pt]{article}

\usepackage{fullpage}
\usepackage[T1]{fontenc}
\usepackage{graphicx}
\usepackage{float}
\usepackage{amsmath}
\usepackage{tabulary}
\usepackage{listings}
\usepackage[spanish]{babel}
\usepackage[utf8]{inputenc}
\usepackage{color}
\usepackage[pdfborder={0 0 0}]{hyperref}
\usepackage{alltt}
\usepackage{moreverb}
\usepackage{enumitem}
\usepackage{array}
\usepackage{amssymb}


% Título principal del documento.`
\begin{document}
\title{	
	\includegraphics[scale=0.8]{images/logo-fiuba.png} \\
	\begin{center}
		\textbf{Trabajo Práctico N$^{\circ}$3} \linebreak
	\end{center}
	\begin{center}
		\begin{large}
			75.29 - Teoría de Algoritmos I \linebreak
			Facultad de Ingeniería de la Universidad de Buenos Aires \linebreak
			1er. Cuatrimestre 2017 \linebreak
		\end{large}
	\end{center} 
}
\author{	Federico Brasburg, \textit{Padrón Nro. 96.653} \\
			\texttt{ federico.brasburg.@gmail.com } \\ [2.5ex]
			Pablo Rodrigo Ciruzzi, \textit{Padrón Nro. 95.748} \\
			\texttt{ p.ciruzzi@hotmail.com } \\ [2.5ex]
			Andrés Otero, \textit{Padrón Nro. 96.604 } \\
			\texttt{ oteroandres95@gmail.com } \\ [2.5ex] \\
\\
		}
\date{23 de junio de 2017}

\maketitle
\thispagestyle{empty}

\pagebreak 

\tableofcontents
\pagebreak

\clearpage
\section{Programación Dinámica}
El problema propuesto es similar al de encontrar el mínimo y máximo elemento en un array de números con la restricción de que el mínimo esté antes que el máximo.
El algoritmo parte de la base de que se compra y se vende el primer día, y guarda el \textit{mínimo actual}. Avanzando en los días, si comprando en el día en el que se está da un beneficio mayor a lo que ya se compró y vendió, cambia y pasa a comprar en este día y a vender en el del \textit{mínimo actual}; si no es así, sigue. Si el precio del día actual es menor al \textit{mínimo actual}, entonces es un potencial día de venta y se actualiza el \textit{mínimo actual}.
 
La ecuación de recurrencia sería algo así:
\begin{equation}
	\label{eq:pg}
	S_{i+1}=\{max\{S_i,k[i+1] - minActual_i\}\}
	\tag{1.1}
\end{equation}
Donde:
\begin{description}
  \item[$\bullet$] En cada $S_i$ se guarda el mínimo y máximo que dan esta diferencia (siendo $S_i$ la diferencia máxima).
  \item[$\bullet$] $k$ es el array con los valores de compra/venta.
\end{description}

\subsection{Cómo correrlo}
Para correr el algoritmo basta con importar la función \texttt{compraVenta} del archivo \texttt{pg.py}, y utilizarla pasándole un array con los números. Dentro del archivo \texttt{pg\_test.py} se puede ver un ejemplo de esto, donde se verifica que el resultado sea correcto comparándolo con el resultado de un algoritmo cuadrático.

\section{Algoritmos Randomizados}
La idea del algoritmo de contracción de Karger es encontrar el corte mínimo en un grafo $G=(V,E)$. Esto es, dos conjuntos no vacíos $A$ y $B$, donde $A \cap B = \varnothing$ y $A \cup B = V$. El tamaño del corte $(A,B)$ se define como el número de aristas $e = (u,v)$ con $u \in A$ y $v \in B$, o viceversa y lo que se busca es que éste sea mínimo.

Este algoritmo es un caso de randomización de tipo \textbf{Monte Carlo}, ya que tiene una probabilidad de que el mismo falle en encontrar la solución correcta.

El algoritmo de contracción en sí consiste en elegir una arista $e=(u,v)$ al azar, y “fusionar” los vértices creando un \textit{supernodo}, el cual tiene tiene todas las aristas de $u$ y $v$ (Salvo la/s arista/s entre ellos). Es importante recalcar que el grafo debe permitir múltiples aristas entre dos vértices, ya que al fusionar es importante que se mantengan la cantidad de aristas (Nuevamente, salvo las que son entre $u$ y $v$). Este proceso se debe repetir hasta que el grafo sólo conste de 2 vértices (Notar que en cada ``iteración'' eliminamos un nodo), los cuales se corresponderán con el corte $(A,B)$, y la cantidad de aristas en el grafo será el tamaño del mismo.

Lo más interesante de esto es que su probabilidad de falla es relativamente alta, pero puede ser reducida ampliamente mediante múltiples (en cantidad polinomial) corridas. Yendo a los números, la probabilidad de que el algoritmo sea correcto con una única corrida es de al menos ${\binom{n}{2}}^{-1}$ (con $n = |V|$), lo cual para un $n$ grande es un número muy chico (Es decir, su probabilidad de falla es como mucho $1 - {\binom{n}{2}}^{-1}$). Pero si el mismo se corre $\binom{n}{2}$ veces, se reduce a que se puede fallar en encontrar el corte mínimo con una probabilidad:
\begin{equation}
	\label{eq:prob}
	(1 - {\binom{n}{2}}^{-1})^{\binom{n}{2}} \leq \frac{1}{e} \tag{2.1}
\end{equation}

Aún más, si se corre $\binom{n}{2}*\ln n$ veces, la probabilidad en \ref{eq:prob} desciende a $e^{-\ln n} = \frac{1}{n}$, lo cual es más que aceptable para un $n$ grande.

\subsection{Cómo correrlo}
Correr el algoritmo con una instancia del grafo con $n$ vértices y $2*n$ aristas es tan simple como correr \texttt{python karger.py n} desde la carpeta \texttt{src} del proyecto.

\section{Algoritmos Aproximados}
El objetivo del algoritmo es resolver el problema de optimización de Subset Sum: dado un set de enteros positivos y $t$ un entero positivo, encontrar la suma más grande de enteros del set que sea menor a $t$.

La idea del algoritmo aproximado de tiempo polinomial viene del algoritmo que lo resuelve de manera exacta, que lo hace en tiempo exponencial. Este mismo surge de una suerte de fuerza bruta, es decir, de calcular la suma de todos los subsets y luego elegir el más cercano a $t$. Pero el algoritmo exacto lo hace de manera más inteligente, ya que para calcular la suma de todos los subsets ${x_1,...,x_i}$, utiliza la suma de todos los subsets ${x_1,...,x_{i-1}}$, y también si se da cuenta que la suma de un cierto subset da mas que $t$, lo elimina. En resumen, el algoritmo exacto toma un subset $S={x_1,...,x_n}$ y el entero $t$, luego va calculando $L_i$ que es la lista de todos los subsets de ${x_1,...,x_i}$ que no superan $t$, y termina devolviendo el valor más alto de $L_n$.

Este algoritmo es exacto pero muy costoso. Buscando una solución lo suficientemente buena y menos costosa podemos usar el aproximado de tiempo polinomial. Este algoritmo ataca el problema de que calcular $L_i$ en cada paso es muy costoso, y lo resuelve recortando (\textit{trimming}) cada lista $L$ luego de crearla. Usa la idea de que si hay 2 valores muy cercanos en la lista no vale la pena mantenerlos a ambos. Utilizando un parámetro $\delta$ de recorte, recorta todos los elementos $y$ tal que existe $z$ en la $L'$ recortada de manera que:
\begin{equation}
	\label{eq:delta}
	\dfrac{y} {1+\delta} \leq z \leq y \tag{3.1}
\end{equation}

Este método se basa en que se pueden quitar muchos elementos sin que ellos queden sin ser representados en la lista recortada. El metodo de recortar la lista $L$ en $\Theta(|L|)$, es usando la lista ordenada, donde agarra el primer elemento y luego va agregando los elementos más grandes mientras no cumpla \ref{eq:delta}. Finalmente, la lista es $L'$.

En resumen, el algoritmo polinómico funciona igual que el exacto pero utiliza el algoritmo de recorte luego de calcular $L_i$, que se calcula utilizando un parámetro de aproximación $0 < \epsilon < 1$ que se usa para calcular $\delta = \dfrac{\epsilon} {2n}$ siendo $n=|S|$.

\subsection{Cómo correrlo}
Se puede crear un problema aleatorio con el método \texttt{generar\_problema\_aleatorio} de la clase \texttt{SubsetSum}, pasando el nombre del archivo a generar, el tamaño $n$ del subset y dos enteros entre los cuales se generan los valores del subset.

Para resolver el problema, se puede utilizar el método \texttt{resolver\_problema} de la misma clase, pasando el nombre del archivo generado, un $t$ y un parametro $\epsilon$ de aproximación.

\pagebreak

\newpage
\section{Código}
\lstset{
	language=Python, columns=flexible, breaklines=true, frame=single, title=creador\_grafos.py
}
\lstinputlisting{../src/creador_grafos.py}

\lstset{ title=grafo.py }
\lstinputlisting{../src/grafo.py}

\lstset{ title=karger.py }
\lstinputlisting{../src/karger.py}

\lstset{ title=parser.py }
\lstinputlisting{../src/parser.py}

\lstset{ title=pg.py }
\lstinputlisting{../src/pg.py}

\lstset{ title=pg\_test.py }
\lstinputlisting{../src/pg_test.py}

\lstset{ title=subset\_sum.py }
\lstinputlisting{../src/subset_sum.py}


\end{document}